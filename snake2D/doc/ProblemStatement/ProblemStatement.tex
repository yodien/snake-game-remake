\documentclass[11pt, oneside]{article}   	% use "amsart" instead of "article" for AMSLaTeX format
\usepackage{geometry}                		% See geometry.pdf to learn the layout options. There are lots.
\geometry{letterpaper}                   		% ... or a4paper or a5paper or ... 
%\geometry{landscape}                		% Activate for rotated page geometry
%\usepackage[parfill]{parskip}    		% Activate to begin paragraphs with an empty line rather than an indent
\usepackage{graphicx}				% Use pdf, png, jpg, or eps§ with pdflatex; use eps in DVI mode
								% TeX will automatically convert eps --> pdf in pdflatex		
\usepackage{amssymb}

\usepackage{fancyhdr}
\pagestyle{fancy}

\title{SE 3XA3: Problem Statement\\Title of Project}


%\begin{header}

%\end{header}
\begin{document}
%\maketitle
\marginpar{}
\section*{authors}
Team \#: 302, Team Name: 404
		\\ Student: Shunbo Cui	cuis13
		\\ Student: Xiangxin Kong	kongx9
		\\ Student: Shuo Zhang	zhans18
\section*{Problem Statement}
\subsection*{What problem are you trying to solve?}

\newcommand*\apos{\textsc{\char13}}

Snake game is a very popular program that is written in many different languages. The original and classic concepts started in 1976, and there are over 300 similar versions for IOS now. However, some important functions are absent in this game, such as ranking list and replay option. Also, the choices of maps for this game are limited and the original design for the interface is rough and simple. We are going to reconstruct the user interface,  to enhance user experience. More gaming systems such as utility items can be introduced to the game to improve the complexity of the game.

\subsection*{Why is this an important problem?}

Increasingly, computer games become a means of entertainment. As game developers, we are trying to create a fun and innovative game for the general public to enjoy. Teens will be more willing to play games that look cool and fun. But, The underlying mechanism of this snake game is very plain. By increasing the complexity and complicacy, we can bring unexpected interest and abundance to the game which promotes the adhesiveness of users.

\subsection*{What is the context of the problem you are solving?}

Our reforged Snake will be free to all general public and open source to all developer who has an interest. The game is based on Java language and capable of most PC systems. It is a lightweight game that can be played by all ages, especially teenagers. They are likely to be interested in this game for relaxing in leisure time. Any player or developer is welcomed to provide their feedback and suggestions to our project.
\end{document}
