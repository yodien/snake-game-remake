\documentclass[12pt, titlepage]{article}

\usepackage{amssymb}
\usepackage{amstext}
\usepackage{amsfonts}
\usepackage{amsthm}
\usepackage{amsmath}
\usepackage{changepage}
\usepackage{enumerate}
\usepackage{fancyhdr}
\usepackage[margin=1in]{geometry}

\usepackage{fullpage}
\usepackage[round]{natbib}
\usepackage{multirow}
\usepackage{booktabs}
\usepackage{tabularx}
\usepackage{graphicx}
\usepackage{float}
\usepackage{hyperref}
\hypersetup{
    colorlinks,
    citecolor=black,
    filecolor=black,
    linkcolor=red,
    urlcolor=blue
}
\usepackage[round]{natbib}

\newcounter{acnum}
\newcommand{\actheacnum}{AC\theacnum}
\newcommand{\acref}[1]{AC\ref{#1}}

\newcounter{ucnum}
\newcommand{\uctheucnum}{UC\theucnum}
\newcommand{\uref}[1]{UC\ref{#1}}

\newcounter{mnum}
\newcommand{\mthemnum}{M\themnum}
\newcommand{\mref}[1]{M\ref{#1}}

\title{SE 3XA3: Module Interface Specification\\\textbf{Snake Game Remake}}

\author{Team \#, 302, Team Name: 404
		\\ Student: Shunbo Cui	    cuis13
		\\ Student: Xiangxin Kong	kongx9
		\\ Student: Shuo Zhang	    zhans18
}

\date{March 13, 2020}

\begin{document}

\maketitle

\begin{table}[hp]
\caption{Revision History} \label{TblRevisionHistory}
\begin{tabularx}{\textwidth}{llX}
\toprule
\textbf{Date} & \textbf{Members} & \textbf{Change}\\
\midrule
March 12, 2020 & Shuo Zhang, Xiangxin Kong, Shunbo Cui & Create the MIS\\
... & ... & ...\\
\bottomrule
\end{tabularx}
\end{table}

\newpage
\section {Map Module}

\subsection*{Module}

Map

\subsection* {Uses}

None

\subsection* {Syntax}

\subsubsection* {Exported Constants}


\textbf{map1} : \{(227, 351, 16, 16),
 (243, 351, 16, 16),
 (259, 351, 16, 16),
 (275, 351, 16, 16),
 (291, 351, 16, 16),
 (307, 351, 16, 16),
 (323, 351, 16, 16),
 (339, 351, 16, 16),
 (355, 351, 16, 16),
 (371, 351, 16, 16),
 (387, 351, 16, 16),
 (403, 351, 16, 16),
 (419, 351, 16, 16),
 (435, 351, 16, 16),
 (451, 351, 16, 16),
 (819, 351, 16, 16),
 (803, 351, 16, 16),
 (787, 351, 16, 16),
 (771, 351, 16, 16),
 (755, 351, 16, 16),
 (739, 351, 16, 16),
 (723, 351, 16, 16),
 (707, 351, 16, 16),
 (691, 351, 16, 16),
 (675, 351, 16, 16),
 (659, 351, 16, 16),
 (643, 351, 16, 16),
 (627, 351, 16, 16),
 (611, 351, 16, 16),
 (595, 351, 16, 16),
 (515, 127, 16, 16),
 (515, 143, 16, 16),
 (515, 159, 16, 16),
 (515, 175, 16, 16),
 (515, 191, 16, 16),
(515, 207, 16, 16),
 (515, 223, 16, 16),
 (515, 239, 16, 16),
 (515, 255, 16, 16),
 (515, 271, 16, 16),
 (515, 575, 16, 16),
 (515, 559, 16, 16),
 (515, 543, 16, 16),
 (515, 527, 16, 16),
 (515, 511, 16, 16),
 (515, 495, 16, 16),
 (515, 479, 16, 16),
 (515, 463, 16, 16),
 (515, 447, 16, 16),
 (515, 431, 16, 16)\} \\\\
 \textbf{map2} : (227, 351, 16, 16),
 (243, 351, 16, 16),
 (259, 351, 16, 16),
 (275, 351, 16, 16),
 (291, 351, 16, 16),
 (819, 351, 16, 16),
 (803, 351, 16, 16),
 (787, 351, 16, 16),
 (771, 351, 16, 16),
 (755, 351, 16, 16),
 (483, 127, 16, 16),
 (483, 143, 16, 16),
 (483, 159, 16, 16),
 (483, 175, 16, 16),
 (483, 191, 16, 16),
 (483, 207, 16, 16),
 (483, 223, 16, 16),
 (483, 239, 16, 16),
 (483, 255, 16, 16),
 (483, 271, 16, 16),
 (483, 287, 16, 16),
 (483, 303, 16, 16),
 (483, 319, 16, 16),
 (483, 335, 16, 16),
 (483, 351, 16, 16),
 (483, 367, 16, 16),
 (483, 383, 16, 16),
 (483, 399, 16, 16),
 (483, 415, 16, 16),
 (483, 431, 16, 16),
 (483, 447, 16, 16),
 (483, 463, 16, 16),
 (483, 479, 16, 16),
 (547, 575, 16, 16),
 (547, 559, 16, 16),
 (547, 543, 16, 16),
 (547, 527, 16, 16),
 (547, 511, 16, 16),
 (547, 495, 16, 16),
 (547, 479, 16, 16),
 (547, 463, 16, 16),
 (547, 447, 16, 16),
 (547, 431, 16, 16),
 (547, 415, 16, 16),
 (547, 399, 16, 16),
 (547, 383, 16, 16),
 (547, 367, 16, 16),
 (547, 351, 16, 16),
 (547, 335, 16, 16),
 (547, 319, 16, 16),
 (547, 303, 16, 16),
 (547, 287, 16, 16),
 (547, 271, 16, 16),
 (547, 255, 16, 16),
 (547, 239, 16, 16),
 (547, 223, 16, 16)\\\\
 


\subsubsection* {Exported Types}

None

\subsubsection* {Exported Access Programs}

None

\subsection* {Semantics}

\subsubsection* {State Variables}

None

\subsubsection* {State Invariant}

None

\section{Snake Module}

Snake

\subsection* {Uses}

None

\subsection* {Syntax}

\subsubsection* {Exported Constants}

None

\subsubsection* {Exported Types}

None

\subsubsection* {Exported Access Programs}

\begin{tabular}{| l | l | l | l |}
\hline
\textbf{Routine name} & \textbf{In} & \textbf{Out} & \textbf{Exceptions}\\
\hline
new snake &  & snake & \\
\hline
setDirectione & $\mathbb{Z}$ &  & \\
\hline
eat &   &  & \\
\hline
move &   &  & \\
\hline
getSnakeBody &  & set of ($\mathbb{R}, \mathbb{R}, \mathbb{R}, \mathbb{R}$)  & \\
\hline
getLength &  & $\mathbb{Z}$  & \\
\hline
getDirection &  & $\mathbb{Z}$  & \\
\hline
getHead &  & ($\mathbb{R}, \mathbb{R}, \mathbb{R}, \mathbb{R}$)  & \\
\hline
\end{tabular}


\subsection* {Semantics}

\subsubsection* {State Variables}

snakebody: set of ($\mathbb{R}, \mathbb{R}, \mathbb{R}, \mathbb{R}$)\\
direction: {$\mathbb{Z}$}\\

\subsubsection* {State Invariant}

DEFAULT\_SNAKE\_LENGTH = 5\\
DEFAULT\_SNAKE\_DIRECTION = 3

\subsubsection* {Assumptions}
\begin{itemize}
\item The constructor snake is called for each object instance before any other
access routine is called for that object.  The constructor cannot be called on
an existing object.
\item direction can only be 0, 1, 2 or 3.
\end{itemize}

\subsubsection* {Access Routine Semantics}

snake():
\begin{itemize}
\item transition: direction := DEFAULT\_SNAKE\_DIRECTION, snakebody := ($\forall i : 0 \leq i \leq$ DEFAULT\_SNAKE\_LENGTH $|$ (355 - i * 16, 191, 16, 16) $\in$ snakebody)
\item output: $\mathit{out} := \mathit{self}$
\item exception: None
\end{itemize}

\noindent setDirection(dir):
\begin{itemize}
\item transition: ((direction $\geq$ 3 $\land$ dir $<$ 3) $\lor$ (direction $\leq$ 2 $\land$ dir $>$ 2)) $\implies$ direction := dir
\item ouput := None
\item exception: None\\
\end{itemize}

\noindent eat():
\begin{itemize}
\item transition: snakebody := snakebody + last ($\mathbb{R}, \mathbb{R}, \mathbb{R}, \mathbb{R}$) element of snakebody before the move [$\leftarrow$ \textit{this element is cut off from the snakebody after every move() operation}]
\item ouput := None
\item exception: None\\
\end{itemize}

\noindent getSnakeBody():
\begin{itemize}
\item transition: None
\item ouput := snakebody
\item exception: None\\
\end{itemize}

\noindent getLength():
\begin{itemize}
\item transition: None
\item ouput := size of snakebody
\item exception: None\\
\end{itemize}

\noindent getDirection():
\begin{itemize}
\item transition: None
\item ouput := direction
\item exception: None\\
\end{itemize}

\noindent getHead():
\begin{itemize}
\item transition: None
\item ouput := first ($\mathbb{R}, \mathbb{R}, \mathbb{R}, \mathbb{R}$) element of snakebody
\item exception: None\\
\end{itemize}

\noindent move():
\begin{itemize}
\item transition: ($\forall$ i $\in$ $\mathbb{Z} |$ size of snakebody $\geq$ i $\geq$ 1  : the ith element of snakebody := the (i-1)th element of snakebody ) \textbf{,} ((direction = 1 $\implies$ decreaseY()) $\lor$ (direction = 2 $\implies$ increaseY()) $\lor$ (direction = 3 $\implies$ increaseX()) $\lor$ (direction = 4 $\implies$ decreaseX()))
\item ouput := None
\item exception: None\\
\end{itemize}

\subsection*{Local Functions}
increaseY: set of ($\mathbb{R}, \mathbb{R}, \mathbb{R}, \mathbb{R}) \rightarrow$ set of ($\mathbb{R}, \mathbb{R}, \mathbb{R}, \mathbb{R}$)\\
\noindent increaseY() $\equiv$ 1st element snakebody := ($\mathbb{R}, \mathbb{R}+16, \mathbb{R}, \mathbb{R}$) of 2nd element of snakebody\\\\
decreaseY: set of ($\mathbb{R}, \mathbb{R}, \mathbb{R}, \mathbb{R}) \rightarrow$ set of ($\mathbb{R}, \mathbb{R}, \mathbb{R}, \mathbb{R}$)\\
\noindent decreaseY() $\equiv$ 1st element snakebody := ($\mathbb{R}, \mathbb{R}-16, \mathbb{R}, \mathbb{R}$) of 2nd element of snakebody\\\\
decreaseX: set of ($\mathbb{R}, \mathbb{R}, \mathbb{R}, \mathbb{R}) \rightarrow$ set of ($\mathbb{R}, \mathbb{R}, \mathbb{R}, \mathbb{R}$)\\
\noindent decreaseX() $\equiv$ 1st element snakebody := ($\mathbb{R}-16, \mathbb{R}, \mathbb{R}, \mathbb{R}$) of 2nd element of snakebody\\\\
increaseX: set of ($\mathbb{R}, \mathbb{R}, \mathbb{R}, \mathbb{R}) \rightarrow$ set of ($\mathbb{R}, \mathbb{R}, \mathbb{R}, \mathbb{R}$)\\
\noindent increaseX() $\equiv$ 1st element snakebody := ($\mathbb{R}+16, \mathbb{R}, \mathbb{R}, \mathbb{R}$) of 2nd element of snakebody\\

\section{Main Module}

Main Screen

\subsection* {Uses}

Game board

\subsection* {Syntax}

\subsubsection* {Exported Constants}

None

\subsubsection* {Exported Types}

None

\subsubsection* {Exported Access Programs}

\begin{tabular}{| l | l | l | l |}
\hline
\textbf{Routine name} & \textbf{In} & \textbf{Out} & \textbf{Exceptions}\\
\hline
Main &  &  &\\
\hline
new MainScreen &  & MainScreen & \\
\hline
\end{tabular}

\subsection* {Semantics}

\subsubsection* {State Variables}

\textit{gameboard} $\in$ GameBoardPanel

\subsubsection* {Environment Variables}

keyboard button: up, down, left ,right, space, esc\\
mouse: leftclick
\subsubsection* {State Invariant}

\textit{levelStrings} = \{ "Easy", "Normal", "Hard" \}

\subsubsection* {Assumptions}
Main() will run before any other programs.

\subsubsection* {Access Routine Semantics}
Main():
\begin{itemize}
\item transition: call Mainscreen()
\item output: None
\item exception: None
\end{itemize}

\noindent Mainscreen():
\begin{itemize}
\item transition: generate a visible window with level selection buttons, a special mode selection button and snake color selection buttons. (select "Easy" $\implies$ call GameBoardPanel(1)) $\lor$ (select "Normal" $\implies$ call GameBoardPanel(2)) $\lor$ (select "Hard" $\implies$ call GameBoardPanel(3)) 
\item ouput := $\mathit{out} := \mathit{self}$
\item exception: None\\
\end{itemize}
%%%%%%%%%%%%%%%%%%%%%%%%%%%%main module end here
%%%%%%%%%%%%%%%%%%%%%%%%%%%%%%sound
\section{Sound Module}

SoundManger

\subsection* {Uses}

None

\subsection* {Syntax}

\subsubsection* {Exported Constants}

None

\subsubsection* {Exported Types}
Files:
bgm\_file = start.wav;
eatfood\_file = food.wav;
eatitem\_file = item.wav;
collide\_file = collision.wav.
\subsubsection* {Exported Access Programs}

\begin{tabular}{| l | l | l | l |}
\hline
\textbf{Routine name} & \textbf{In} & \textbf{Out} & \textbf{Exceptions}\\
\hline
new SoundManger & String & SoundManger &\\
\hline
startSound &  &  & \\
\hline
pauseSound &  &  & \\
\hline
stopSound &  &  & \\
\hline
\end{tabular}

\subsection* {Semantics}

\subsubsection* {State Variables}

None

\subsubsection* {Environment Variables}

\subsubsection* {State Invariant}

None

\subsubsection* {Assumptions}
SoundManger will run before any other programs in this module.

\subsubsection* {Access Routine Semantics}
SoundManger(soundFilePath):
\begin{itemize}
\item transition: Open the sound effect that is stored at path \textit{soundFilePath} and ready to play. 
\item ouput := $\mathit{out} := \mathit{self}$
\item exception: None
\end{itemize}

\noindent startSound():
\begin{itemize}
\item transition: play the sound effect. 
\item ouput : None
\item exception: None\\
\end{itemize}

\noindent pauseSound():
\begin{itemize}
\item transition: suspend the sound effect. 
\item ouput : None
\item exception: None\\
\end{itemize}

\noindent stopSound():
\begin{itemize}
\item transition: terminate the sound effect. 
\item ouput : None
\item exception: None\\
\end{itemize}
%%%%%%%%%%%%%%%%%%%%%%%%%%%%%%%%%%%%

%%%%%%%%%%%%%%%%%%%%
%以下夹在gamboard module 里


%%%%%%%%%%%%%%%%%%



\section{MIS of Food and Item Module}
		\subsection{Interface Syntax}
		\subsubsection{Exported Access Programs}
		\begin{tabular}[pos]{|c|c|c|c|}
			\hline
			\textbf{Name}& \textbf{In} & \textbf{Out} & \textbf{Exceptions} \\ \hline
			generateFood &- & - & - \\ \hline
			getFood & -  &(\mathbb{R}, \mathbb{R})& - \\ \hline
			generateItem & - & - & - \\ \hline
			getItem & - & (\mathbb{R}, \mathbb{R})& - \\ \hline
			
		\end{tabular}
		
		\subsection{Interface Semantics}
		\subsubsection{State Variables}
		food: (\mathbb{R}, \mathbb{R}) - represents the current location of the food\\
		item: (\mathbb{R}, \mathbb{R}) - represents the current location of the item
		
		\subsubsection{Assumptions}
		Other modules always call generateFood() before calling getFood() \\ 
		Other modules always call generateItem() before calling getItem() \\ 
		
		\subsubsection{Access Program Semantics}
		generateFood():
		
        transition: item:=pair of random integers from 0 to 227\\
		\\
		generateItem():
		
        transition: food:=pair of random integers from 0 to 227\\
        \\
		getFood():
		
		Output:= food\\
		\\
		getItem():
		

		Output:=item
		\\
	

%%%%%%%%%%%%%%%%%%%%%%%%%%%%%%%%%%
\section{GameBoard Module}
\subsection* {Uses}
Snake, Food and Item, Sound, Map
\subsection* {Syntax}
\subsubsection* {Exported Constants}
%%%%%%%%%%%%%
		\subsection{Interface Syntax}
		\subsubsection{Exported Access Programs}
		\begin{tabular}[pos]{|c|c|c|c|}
			\hline
			\textbf{Name}& \textbf{In} & \textbf{Out} & \textbf{Exceptions} \\ \hline
			 keyPressed & event & - & - \\ \hline
			 InputManger & event  & - & - \\ \hline
			 checkCollision & -  & $\mathbb{B}$ & - \\ \hline
			 drawMap & -  & - & - \\ \hline
			 drawSnake  & snake  & - & - \\ \hline
			 DrawSnakeFood & food  & - & - \\ \hline
			 DrawStatusbar & -  & - & - \\ \hline
			
		\end{tabular}
		
		
		\subsection{Interface Semantics}
		\subsubsection{State Variables}
		food: ($\mathbb{R}, \mathbb{R}$) - represents the current location of the food\\
		item: ($\mathbb{R}, \mathbb{R}$) - represents the current location of the item\\
		isGameOver:$\mathbb{B}$ - repersents is the game over
		playerScore:$\mathbb{R}$ -repersents the current score
        snake: SNAKE- repersents the current snake
		
		\subsubsection{Assumptions}
		Variables should be set before trying to access them.\\
		If no event is chosen, checkEvent returns a default value 0\\
        If currState is 0, drawInterface does not change\\
        
		\subsubsection{Access Program Semantics}

		
	
InputManger(gb):
\begin{itemize}
\item transition: gameboard := gb
\item output: $\mathit{out} := \mathit{self}$
\item exception: None
\end{itemize}

\noindent keyPressed(e):
\begin{itemize}
\item transition: (e = up button $\implies$ gameBoard.changeSnakeDirection(1)) $\lor$ (e = down button $\implies$ gameBoard.changeSnakeDirection(2)) $\lor$ (e = right button $\implies$ gameBoard.changeSnakeDirection(3)) $\lor$ (e = left button $\implies$ gameBoard.changeSnakeDirection(4)) $\lor$ (e = space $\land$ gameBoard.isGameRunning()=true $\implies$ gameBoard.pauseGame()) $\lor$ (e = space $\land$ gameBoard.isGameRunning()=false $\implies$ gameBoard.startGame()) $\lor$ (e = esc $\implies$ exit the game.)
\item ouput := None
\item exception: None\\
\end{itemize}

%%%%%%%%%
\noindent DrawSnake(a):
\begin{itemize}
\item Input: snake object
\item draws the snake corresponding to the current state to the output window
\item exception: None\\
\end{itemize}

\noindent DrawMap(a):
\begin{itemize}
\item Input: None
\item draws the map corresponding to the map constant to the output windows
\item exception: None\\
\end{itemize}

\noindent DrawSnakeFood(a):
\begin{itemize}
\item Input: food
\item draws the snake food corresponding to the current food state to the map
\item exception: None\\
\end{itemize}


\noindent DrawStatusbar():
\begin{itemize}
\item Input: None
\item Draws the instructions, Show playscore, (Game over $\implies$ "Game over message at center")
\item exception: None\\
\end{itemize}
%%%%%%%%%%


\end{document}


